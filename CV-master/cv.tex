\documentclass[a4paper]{my_cv}
\usepackage{xeCJK}
\usepackage{fontspec, xunicode, xltxtra}  
\setCJKfamilyfont{song}{SimSun}                             %宋体 song  
\newcommand{\song}{\CJKfamily{song}}                        % 宋体   (Windows自带simsun.ttf)  
\setCJKfamilyfont{xs}{NSimSun}                              %新宋体 xs  
\newcommand{\xs}{\CJKfamily{xs}}  
\setCJKfamilyfont{fs}{FangSong_GB2312}                      %仿宋2312 fs  
\newcommand{\fs}{\CJKfamily{fs}}                            %仿宋体 (Windows自带simfs.ttf)  
\setCJKfamilyfont{kai}{KaiTi_GB2312}                        %楷体2312  kai  
\newcommand{\kai}{\CJKfamily{kai}}                            
\setCJKfamilyfont{yh}{Microsoft YaHei}                    %微软雅黑 yh  
\newcommand{\yh}{\CJKfamily{yh}}  
\setCJKfamilyfont{hei}{SimHei}                                    %黑体  hei  
\newcommand{\hei}{\CJKfamily{hei}}                          % 黑体   (Windows自带simhei.ttf)  
\setCJKfamilyfont{msunicode}{Arial Unicode MS}            %Arial Unicode MS: msunicode  
\newcommand{\msunicode}{\CJKfamily{msunicode}}  
\setCJKfamilyfont{li}{LiSu}                                            %隶书  li  
\newcommand{\li}{\CJKfamily{li}}  
\setCJKfamilyfont{yy}{YouYuan}                             %幼圆  yy  
\newcommand{\yy}{\CJKfamily{yy}}  
\setCJKfamilyfont{xm}{MingLiU}                                        %细明体  xm  
\newcommand{\xm}{\CJKfamily{xm}}  
\setCJKfamilyfont{xxm}{PMingLiU}                             %新细明体  xxm  
\newcommand{\xxm}{\CJKfamily{xxm}}  

\setCJKfamilyfont{hwsong}{STSong}                            %华文宋体  hwsong  
\newcommand{\hwsong}{\CJKfamily{hwsong}}  
\setCJKfamilyfont{hwzs}{STZhongsong}                        %华文中宋  hwzs  
\newcommand{\hwzs}{\CJKfamily{hwzs}}  
\setCJKfamilyfont{hwfs}{STFangsong}                            %华文仿宋  hwfs  
\newcommand{\hwfs}{\CJKfamily{hwfs}}  
\setCJKfamilyfont{hwxh}{STXihei}                                %华文细黑  hwxh  
\newcommand{\hwxh}{\CJKfamily{hwxh}}  
\setCJKfamilyfont{hwl}{STLiti}                                        %华文隶书  hwl  
\newcommand{\hwl}{\CJKfamily{hwl}}  
\setCJKfamilyfont{hwxw}{STXinwei}                                %华文新魏  hwxw  
\newcommand{\hwxw}{\CJKfamily{hwxw}}  
\setCJKfamilyfont{hwk}{STKaiti}                                    %华文楷体  hwk  
\newcommand{\hwk}{\CJKfamily{hwk}}  
\setCJKfamilyfont{hwxk}{STXingkai}                            %华文行楷  hwxk  
\newcommand{\hwxk}{\CJKfamily{hwxk}}  
\setCJKfamilyfont{hwcy}{STCaiyun}                                 %华文彩云 hwcy  
\newcommand{\hwcy}{\CJKfamily{hwcy}}  
\setCJKfamilyfont{hwhp}{STHupo}                                 %华文琥珀   hwhp  
\newcommand{\hwhp}{\CJKfamily{hwhp}}  

\setCJKfamilyfont{fzsong}{Simsun (Founder Extended)}     %方正宋体超大字符集   fzsong  
\newcommand{\fzsong}{\CJKfamily{fzsong}}  
\setCJKfamilyfont{fzyao}{FZYaoTi}                                    %方正姚体  fzy  
\newcommand{\fzyao}{\CJKfamily{fzyao}}  
\setCJKfamilyfont{fzshu}{FZShuTi}                                    %方正舒体 fzshu  
\newcommand{\fzshu}{\CJKfamily{fzshu}}  

\setCJKfamilyfont{asong}{Adobe Song Std}                        %Adobe 宋体  asong  
\newcommand{\asong}{\CJKfamily{asong}}  
\setCJKfamilyfont{ahei}{Adobe Heiti Std}                            %Adobe 黑体  ahei  
\newcommand{\ahei}{\CJKfamily{ahei}}  
\setCJKfamilyfont{akai}{Adobe Kaiti Std}                            %Adobe 楷体  akai  
\newcommand{\akai}{\CJKfamily{akai}}  

%------------------------------设置字体大小------------------------%  
\newcommand{\chuhao}{\fontsize{42pt}{\baselineskip}\selectfont}     %初号  
\newcommand{\xiaochuhao}{\fontsize{36pt}{\baselineskip}\selectfont} %小初号  
\newcommand{\yihao}{\fontsize{28pt}{\baselineskip}\selectfont}      %一号  
\newcommand{\erhao}{\fontsize{21pt}{\baselineskip}\selectfont}      %二号  
\newcommand{\xiaoerhao}{\fontsize{18pt}{\baselineskip}\selectfont}  %小二号  
\newcommand{\sanhao}{\fontsize{15.75pt}{\baselineskip}\selectfont}  %三号  
\newcommand{\sihao}{\fontsize{14pt}{\baselineskip}\selectfont}%     四号  
\newcommand{\xiaosihao}{\fontsize{12pt}{\baselineskip}\selectfont}  %小四号  
\newcommand{\wuhao}{\fontsize{10.5pt}{\baselineskip}\selectfont}    %五号  
\newcommand{\xiaowuhao}{\fontsize{9pt}{\baselineskip}\selectfont}   %小五号  
\newcommand{\liuhao}{\fontsize{7.875pt}{\baselineskip}\selectfont}  %六号  
\newcommand{\qihao}{\fontsize{5.25pt}{\baselineskip}\selectfont}    %七号  
\begin{document}

%\addtolength{\bottommargin}{-5cm}
\thispagestyle{empty}	
\pageheader{\ahei 令~}{\hei 平子}
	{+15070614696}
 	{\href{mailto:nikola.n.milev@gmail.com}{nikola.n.milev@\\gmail.com}}
	{ \href{https://github.com/NikolaMilev}{/NikolaMilev}} 	
 	{ \href{https://www.linkedin.com/in/nikola-n-milev/}{/in/nikola-n-milev/}}
	{ Vojislava Ilica, 87, Belgrade }
	{12.20.1998.}

% TODO
% izvali kako da promenis font na neki koji je kul (pogledaj ove tempate?) DONE
% ubaci fotku onu sa strane (vidi kako je u template i da li tikz moze to da uradi) DONE
% izvali kako da podesis kolonu sa strane gde mozes da stavljas stvari (opet, vidi template) DONE
% vidi kako se ukljucuju progress barovi DONE
% vidi kako u sekciji za obrazovanje da ga nateras da bude minimalne sirine te i te
% itd (nadji jos kul ideja)

 
\begin{aside}
\section{\hei \xiaoerhao 语言}
% 3 is max
\bodyfont\skills{{汉语/3}, {英语/2}}
~
~
\section{Programming}
% 3 is max
\skills{{OpenSSL/1.3}, {Flex/1.5}, {Bison/1.5}, {Eclipse/2}, {Intellij IDEA/2}, {Git/2}, {CSS/2}, {HTML/2}, {LaTeX/2}, {GNU Linux/3},{Haskell/2.3},{Python/2.3}}%, {C++/2.5},{Java/3}, {C/3}
\end {aside}
~
~\\
\section{\hei 经历 }
\begin{entrylist}
\entry
    {10/2016~\textemdash \\present}
    {Teaching Associate, Department of Computer Science}
    {Faculty of Mathematics, University of Belgrade}
    {Teaching courses, organising and grading exams. Courses taught: Introduction to programming, Introduction to object oriented programming and Introduction to computer organisation and architecture}
\entry
    {07/2016~\textemdash \\10/2016}
    {Intern}
    {ESDL (Electronics Systems Design Limited), Malta}
    {Implemented a server RaspberryPi with UART communication. Implemented in C, using OpenSSL}
\entry
    {05/2016}
    {Intern}
    {sTech d.o.o. Belgrade, member of UNIQA Group Austria}
    {Worked within three teams in order to get introduced to the system used for processing insurance policies. }
\end{entrylist}
\section{\hei 教育背景}
\begin{entrylist}
  \entry
    {2016~\textemdash \\present}
    {Master's Degree in Computer Science}
    {Faculty of Mathematics, University of Belgrade}
    {Currently learning about machine learning, functional programming, automated reasoning, etc. GPA 10 out of 10.}
  \entry
    {2012~\textemdash \\2016}
    {Bachelor's Degree in Computer Science}
    {Faculty of Mathematics, University of Belgrade}
    {Passed many courses that covered important topics such as algorithms, object oriented programming, Unix system programming, etc. Graduated as one of the best students in the generation. GPA 9.61 out of 10.}
  \entry
    {2008~\textemdash \\2012}
    {High School}
    {Grammar School, Valjevo}
    {Finished with several awards for good students. Was a member of the school choir and took part in various music manifestations.}
\end{entrylist}

\section{\hei 个人荣誉}
\bodyfont
\begin{aplist}
\apentry{2016}{Dositeja scholarship: a scholarship awarded to 800 best students of undergraduate studies in Serbia.}
\apentry{10/2016}{Brand New Engineers Hackathon, team Schwifty, 3rd place.}
\end{aplist}
%\section{Projects}
%\bodyfont
%\begin{aplist}
%\apentry{Origami \\simulator}{A simulation for origami representation in 3D, with basic paper folding allowed. (a team project written in C++ using STL and Qt libraries; implemented the data structure for an origami figure, serialization and several smaller tasks)}
%\end{aplist}

\section{\hei 项目经历}
\setlength{\tabcolsep}{6pt}
\begin{tabularx}{1.07\textwidth}{XX}
\textbf{Minipascal to flowchart:} A small program that compiles a small subset of Pascal into a LaTeX flowchart, written in C++ using Flex and Bison. &
\textbf{Origami simulator:} A group project written in C++ using STL and Qt libraries. Implemented the data structure, serialization and several smaller tasks. \\
\textbf{Turing machine:} A Turing machine simulation, written in Java. &
\textbf{Minesweeper:} An implementation of the game written in Java. \\
\end{tabularx}
~
\section{\hei 兴趣}
\bodyfont
\begin{flushleft} 
\begin{tabularx}{\textwidth}{XXX}

\faBicycle ~Cycling &
\faLifeRing ~~Swimming &
\faUniversity ~~Sightseeing \\
\faMusic ~~~Music &
\faPlane ~~~Travelling &
\faCutlery ~~~Cooking \\
\faKeyboardO ~~Programming &
\faGamepad ~~Video games &
\faTelevision ~~Movies and TV

\end{tabularx}
\end{flushleft}

\end{document} 